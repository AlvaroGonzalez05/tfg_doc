\chapter{Conclusiones y Trabajo Futuro}
\section{Conclusiones}
En este trabajo se ha presentado un sistema de predicción y gestión de la demanda energética para 
la carga de vehículos eléctricos, utilizando un enfoque de modelado híbrido que combina técnicas 
de aprendizaje automático y aprendizaje por refuerzo con memoria, basado en IA generativa.\\

Tanto el optimizador clásico como el sistema de IA generativa han resultado eficaces para gestionar 
la carga de vehículos eléctricos bajo las restricciones planteadas. El enfoque clásico, mediante 
programación lineal, ha ofrecido soluciones óptimas en coste y eficiencia, respetando tanto los 
límites técnicos como las preferencias del usuario. Por su parte, la IA generativa ha destacado por 
su capacidad de adaptación ante escenarios variables, logrando minimizar costes y aprovechar al 
máximo la energía disponible, siempre dentro de los requisitos establecidos.\\

También se ha logrado generar perfiles sintéticos de demanda y disponibilidad de energía con la 
ayuda de técnicas de IA Generativa, lo que ha permitido simular escenarios realistas para la carga 
de vehículos eléctricos.

Finalmente, tras un análisis comparativo de los resultados obtenidos por ambos enfoques, se ha
concluido que, aunque el optimizador clásico garantiza soluciones óptimas bajo restricciones
específicas, el sistema de IA generativa ofrece una mayor flexibilidad y adaptabilidad a cambios
en las condiciones de carga y disponibilidad de energía. Esto lo convierte en una herramienta valiosa
para la planificación y operación de sistemas energéticos cada vez más complejos, especialmente en
el contexto de la movilidad eléctrica y la integración de energías renovables.\\

\section{Próximos Pasos}
Aunque el sistema desarrollado ha demostrado ser eficaz y cumple con los objetivos planteados,
existen varias áreas de mejora y expansión que podrían explorarse en el futuro:
\begin{itemize}
    \item \textbf{Introducción de algoritmos de clustering:} Incorporar técnicas como 
    \textit{k-means} \cite{lloyd1982kmeans} o \textit{k-nearest neighbors} \cite{altman1992knn} para 
    identificar y clasificar perfiles de demanda y patrones de carga, permitiendo una gestión 
    más personalizada y eficiente de los recursos energéticos.
    \item \textbf{Mejora de la política de control mediante métodos Actor-Critic:} Implementar 
    enfoques avanzados de aprendizaje por refuerzo, como métodos Actor-Critic o políticas de 
    gradiente, que permitan asignar recompensas suaves (\textit{soft rewards}) en función de la 
    dirección y calidad de las acciones tomadas por el sistema, facilitando un aprendizaje más fino 
    y adaptativo.
    \item \textbf{Capacidades predictivas avanzadas:} Integrar modelos de predicción para anticipar 
    la demanda energética y la disponibilidad de recursos, mejorando la planificación y la toma de 
    decisiones en tiempo real.
    \item \textbf{Ampliación de la memoria y el \textit{replay buffer}:} Aumentar la capacidad de 
    memoria del sistema, permitiendo el acceso a experiencias pasadas relevantes, como datos de 
    fechas o situaciones similares de años anteriores, para enriquecer el aprendizaje y la toma de 
    decisiones.
    \item \textbf{Integración de energías renovables en el consumo no gestionable:} Considerar la 
    aportación de fuentes renovables en la parte de la demanda no gestionable, incrementando la 
    sostenibilidad y la eficiencia global del sistema.
\end{itemize}