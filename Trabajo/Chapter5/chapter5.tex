\chapter{Resultados}
\label{chapter5}
En el capítulo anterior se ha visto cómo se desenvolían los distintos algoritmos de optimización 
intentando resolver el problema planteado en el trabajo. A lo largo de este capítulo, se va a hacer
un análisis conjunto de los resultados obtenidos por separado para cada uno de los algoritmos. Se 
hará especial hincapié en tres comparaciones fundamentales para el proyecto: coste, energía y 
rendimiento. Adicionalmente, como derivado natural de la evaluación de los costes y el consmo 
energético, se analizará también la eficiencia de cada uno de los algoritmos.

Para facilitar la comparación entre la optimización clásica y el agente DQN-LSTM, en cada sección 
se presentarán los resultados de ambos métodos en paralelo, utilizando tablas comparativas. De este 
modo, se puede observar de forma directa las diferencias y ventajas de cada enfoque para cada perfil 
analizado. Además, se puede calcular el porcentaje de diferencia (mejora o empeoramiento) de DQN-LSTM 
respecto a la optimización clásica para cada métrica relevante, lo que permite cuantificar el impacto 
de la inteligencia artificial frente a los métodos tradicionales.

\section{Comparativa de Costes}
La primera comparativa que se va a realizar es la de los costes medios diarios por perfil,
tanto para la optimización clásica como para el agente DQN-LSTM. Para ello, se muestran en la
Tabla~\ref{tab:costes_algoritmos} los costes medios diarios para cada perfil:
\begin{table}[H]
    \centering
    \resizebox{\textwidth}{!}{
    \begin{tabular}{lccc}
        \toprule
        \textbf{Perfil} & \textbf{Costes - Clásica} (EUR) & \textbf{Costes - DQN-LSTM} (EUR) & \textbf{Reducción Costes(\%)} \\
        \midrule
        retired     & 27.14 & 1.64 & 93.96 \\
        night\_owl  & 27.57 & 6.21 & 77.47 \\
        flexible    & 27.69 & 2.93 & 89.41 \\
        traveller   & 27.23 & 7.09 & 73.96 \\
        worker      & 27.35 & 5.04 & 81.57 \\
        \midrule
        \textbf{Media} & \textbf{27.40} & \textbf{4.18} & \textbf{84.74} \\
        \bottomrule
    \end{tabular}
    }
    \caption{Comparativa de costes medios diarios por perfil entre la optimización clásica y el 
    agente DQN-LSTM, junto con el porcentaje de reducción de coste.}
    \label{tab:costes_algoritmos}
\end{table}

"[pongo un gráfico en vez de la tabla??]"\\

En esta primera ronda, se puede observar fácilmente como la IA ha conseguido reducir unos datos
de coste, que ya de por sí no eran extremadamente elevados con la optimización clásica, a unas 
cifras muy reducidas. En concreto, el agente DQN-LSTM ha conseguido reducir los costes
medios diarios en un 84.74\% respecto a la optimización clásica, lo que supone una mejora
significativa en términos de eficiencia económica. Esta reducción es especialmente notable en
el perfil \textit{retired}, donde se ha alcanzado una reducción del 93.96\%. Esto se podría deber 
a que este perfil tiene un consumo energético más predecible y estable, lo que permite al
agente DQN-LSTM optimizar mejor los tiempos de carga y los costes asociados.\\

Por otro lado, el perfil \textit{night\_owl} ha mostrado una reducción del 77.47\%, lo que indica que,
aunque la optimización clásica ya era eficiente, el agente DQN-LSTM ha logrado mejorar aún más
la gestión de los tiempos de carga nocturna, aprovechando las tarifas más bajas.\\

Los perfiles \textit{flexible} y \textit{traveller} también han mostrado reducciones significativas,
con un 89.41\% y un 73.96\% respectivamente. Esto sugiere que, aunque estos perfiles pueden tener
un consumo mucho más variable, el agente DQN-LSTM ha logrado adaptarse a sus necesidades específicas,
optimizando los tiempos de carga y reduciendo los costes asociados.\\

Por último, el perfil \textit{worker} ha mostrado una reducción del 81.57\%, ligeramente por debajo
de la media, lo que indica que el agente DQN-LSTM, de nuevo, ha logrado una mejora considerable 
para este perfil también. Esto puede deberse a que los trabajadores suelen tener horarios más
predecibles, lo que permite al agente optimizar mejor los tiempos de carga y los costes asociados.\\

Cerrando este primer análisis, se puede concluir que, en este estudio, la optimización clásica queda
muy por detrás económicamente de la optimización mediante el agente DQN-LSTM, que ha conseguido 
incurrir en casi un 85\% menos de costes diarios.

\section{Comparativa de Energía}
De la misma manera que con el coste, se va a realizar una comparativa de la energía consumida por 
cada uno de los algoritmos. Para ello, se muestran en la Tabla~\ref{tab:energia_algoritmos} los 
consumos medios diarios para cada perfil, diferenciando entre la optimización clásica y el agente 
DQN-LSTM.
\begin{table}[H]
    \centering
    \resizebox{\textwidth}{!}{
    \begin{tabular}{lccc}
        \toprule
        \textbf{Perfil} & \textbf{Energía - Clásica} (kWh) & \textbf{Energía - DQN-LSTM} (kWh) & \textbf{Reducción Energía (\%)} \\
        \midrule
        retired     & 244.08 & 63.28 & 74.08 \\
        night\_owl  & 244.53 & 268.06 & -9.62 \\
        flexible    & 245.02 & 120.07 & 50.99 \\
        traveller   & 242.74 & 325.07 & -33.97 \\
        worker      & 244.08 & 209.68 & 14.12 \\
        \midrule
        \textbf{Media} & \textbf{244.09} & \textbf{197.23} & \textbf{19.21} \\
        \bottomrule
    \end{tabular}
    }
    \caption{Comparativa de energía media diaria por perfil entre la optimización clásica y el agente DQN-LSTM, junto con el porcentaje de reducción de consumo energético.}
    \label{tab:energia_algoritmos}
\end{table}

En esta comparativa se busca que el gestor, ya sea el agente DQN-LSTM o la optimización clásica,
consuma la menor cantidad de energía posible, ya que esto se traduce directamente en una reducción 
de costes en la carga. En este caso, aunque la IA ha conseguido reducir la energía consumida casi 
en un 20\% de media, no lo ha hecho de forma uniforme para todos los perfiles.\\

La mayor diferencia entre consumos se encuentra con el perfil \textit{retired}, donde la red 
neuronal consigue una reducción del 74.08\% respecto a la optimización clásica, una cifra 
tremendamente elevada. Esto puede deberse a que este perfil tiene un consumo energético mucho más 
estable y predecible, lo que permite al agente DQN-LSTM aprovechar su entorno y optimizar los tiempos
de carga para reducir el consumo energético.\\

Por otra parte, el peril night\_owl ha mostrado un aumento del 9.62\% en el consumo energético,
lo que indica que la optimización clásica ya era eficiente, y el agente DQN-LSTM no ha logrado
ni siquiera igualar ese consumo. Esto puede deberse a que este perfil tiene un consumo energético
diurno, y puede haberse visto afectado por la rareza de esta característica, ya que la mayoría del 
resto de perfiles tienen un consumo nocturno. En este caso, puede ser que la cpaacidad de memoria
de la inteligencia artificial haya sido un detrimento para su rendimiento en este perfil.\\

El perfil \textit{flexible} ha mostrado una reducción del 50.99\%, lo que muestra que el agente 
DQN-LSTM se adapta mucho mejor a la flexibilidad de carga de este perfil, contrastando con la 
rigidez de la optimización clásica. Esto sugiere que el agente ha logrado optimizar los tiempos de
carga y reducir el consumo energético, aprovechando las horas de menor demanda.\\

El perfil \textit{traveller} ha mostrado un aumento del 33.97\% en el consumo energético, lo que
indica que la optimización clásica ya era eficiente, y el agente DQN-LSTM de nuevo no ha logrado
si quiera igualar ese consumo.\\

Por último, el perfil \textit{worker} ha mostrado una reducción del 14.12\%, lo que indica que
el agente DQN-LSTM ha logrado una mejora considerable en la gestión de los tiempos de carga y el
consumo energético en el perfil más representativo, aunque no tan significativa como en el caso 
del perfil \textit{retired}. Esto puede deberse a que los trabajadores suelen tener horarios más
predecibles, lo que permite al agente optimizar mejor los tiempos de carga y reducir el consumo
energético.\\

La conclusión de esta comparativa no es del todo clara, ni positiva hacia el agente DQN-LSTM.
Aunque ha conseguido reducir el consumo energético - de media - en un 19.21\%, lo que supone una 
mejora considerable, no ha logrado igualar los resultados de la optimización clásica en todos los
perfiles. De hecho, hay un perfil en el que ha aumentado considerablemente el consumo energético 
en comparación con la gestión realizada por el algoritmo clásico. Por tanto, aunque la IA ha
conseguido una mejora significativa en la reducción de costes, no ha logrado una mejora uniforme en
el consumo energético.

\section{Comparativa de Eficiencia}
Siguiendo la misma línea que en las comparativas anteriores, se va a realizar una comparativa de la 
eficiencia de cada uno de los algoritmos. Para ello, se muestran en la Tabla~\ref{tab:eficiencia_algoritmos} 
las eficiencias medias diarias para cada perfil, diferenciando entre la optimización clásica y el 
agente DQN-LSTM. La eficiencia se calcula como el cociente entre la energía consumida y el coste, 
es decir, \( \text{Eficiencia} = \frac{\text{Energía}}{\text{Coste}} \).
\begin{table}[H]
    \centering
    \resizebox{\textwidth}{!}{
    \begin{tabular}{lccc}
        \toprule
        \textbf{Perfil} & \textbf{Eficiencia - Clásica} (kWh/EUR) & \textbf{Eficiencia - DQN-LSTM} (kWh/EUR) & \textbf{Mejora Eficiencia(\%)} \\
        \midrule
        retired     & 8.99  & 37.14 & 313.16 \\
        night\_owl  & 8.87  & 43.07 & 385.77 \\
        flexible    & 8.85  & 40.33 & 355.99 \\
        traveller   & 8.92  & 45.93 & 415.13 \\
        worker      & 8.92  & 41.63 & 366.77 \\
        \midrule
        \textbf{Media} & \textbf{8.91} & \textbf{41.62} & \textbf{367.18} \\
        \bottomrule
    \end{tabular}
    }
    \caption{Comparativa de la eficiencia diaria (kWh/EUR) por perfil entre la optimización clásica y el agente DQN-LSTM, junto con el porcentaje de mejora en eficiencia.}
    \label{tab:eficiencia_algoritmos}
\end{table}

La eficiencia es una métrica que combina el coste y el consumo energético, y dado la disparidad en 
los resultados de las dos comparativas anteriores, se espera que la eficiencia sea una métrica más
representativa del rendimiento de cada uno de los algoritmos. Naturalmente, al representar esta
medida, esencialmente, la cantidad de kWh que usados por cada euro invertido, cuanto mayor sea el
valor, mejor.\\

En este caso sí, la red neuronal ha pasado por encima de la optimización clásica en todos los perfiles.
Tanto es así, que no merece la pena el análisis de cada perfil individualmente, ya que la mejora es
de tal magnitud que se puede considerar como una gran mejora generalizada. En concreto, el agente 
DQN-LSTM ha conseguido una mejora de la eficiencia del 367.18\% respecto a la optimización clásica,
una diferencia abismal que demuestra la capacidad de la IA para optimizar los tiempos de carga y reducir
los costes asociados.\\

Si bien es cierto que esta métrica deriva de las dos anteriores, y por tanto, no es una métrica
independiente. Por ello, podría estar sesgada por los resultados de las comparativas de coste y 
energía. Sin embargo, la magnitud de la mejora es tal, que puede considerarse una mejora 
significativa y representativa del rendimiento del agente DQN-LSTM frente a la optimización 
clásica.

\section{Comparativa de Rendimiento}
Finalmente, se va a realizar una comparativa del rendimiento de cada uno de los algoritmos.
Para ello, se muestran en la Tabla~\ref{tab:rendimiento_algoritmos} los rendimientos medios para 
ambos algoritmos. El rendimiento en este caso se va a basar en el tiempo de ejecución y de decisión/
inferencia de cada uno de los algoritmos. Se ha medido el tiempo medio de ejecución/entrenamiento de
cada uno de los algoritmos, así como el tiempo medio de decisión/inferencia.\\

En general, la optimización clásica suele requerir menos tiempo de cálculo para problemas 
individuales, pero no es escalable ni adaptable a cambios dinámicos. Por otro lado, el agente DQN-
LSTM requiere un tiempo de entrenamiento inicial mayor, pero una vez entrenado, la inferencia es 
extremadamente rápida y permite adaptarse a nuevas situaciones en tiempo real.

\begin{table}[H]
    \centering
    \begin{tabular}{lccc}
        \toprule
        \textbf{Perfil} & \textbf{Tiempo total (min)} & \textbf{Tiempo de Solución (ms)} & \textbf{Método} \\
        \midrule
        worker      & 0.13\footnotemark[1] & 9.50\footnotemark[2] & Clásica \\
        flexible    & 0.13\footnotemark[1] & 9.50\footnotemark[2] & Clásica \\
        retired     & 0.13\footnotemark[1] & 9.50\footnotemark[2] & Clásica \\
        traveller   & 0.13\footnotemark[1] & 9.50\footnotemark[2] & Clásica \\
        night\_owl  & 0.13\footnotemark[1] & 9.50\footnotemark[2] & Clásica \\
        \midrule
        worker      & 3.98 & 0.174 & DQN-LSTM \\
        flexible    & 4.39 & 0.294 & DQN-LSTM \\
        retired     & 4.29 & 0.301 & DQN-LSTM \\
        traveller   & 4.45 & 0.309 & DQN-LSTM \\
        night\_owl  & 4.37 & 0.306 & DQN-LSTM \\
        \bottomrule
    \end{tabular}
    \caption{Comparativa de tiempos de entrenamiento/optimización total y de inferencia/decisión 
    por perfil y método. Para la optimización clásica, el tiempo total corresponde al tiempo total 
    de optimización (8.05 s $\approx$ 0.13 min) y el tiempo de inferencia al tiempo resolviendo 
    modelos individuales (0.57 s / 60 $\approx$ 9.50 ms por perfil).}
\end{table}

\footnotetext[1]{Tiempo total de optimización clásica para todos los perfiles.}
\footnotetext[2]{Tiempo de inferencia de la optimización clásica para cada perfil.}

El análisis de rendimiento computacional, es decir, cuanto tiempo requiere cada programa para 
resolver el problema para cada uno de los perfiles, es un aspecto fundamental del trabajo. De cara 
a la implementación de un sistema real, es necesario que el tiempo de respuesta y las demandas 
computacionales sean lo más bajas posibles, para que el sistema pueda adaptarse a las
necesidades de los usuarios en tiempo real, sin pérdidas en la calidad del servicio ni requisitos 
de hardware excesivos.\\

Por la naturaleza de los algoritmos, es necesario dividir la comparación en dos partes: el tiempo
que requiere cada algooritmo para optimizar el problema como tal, con todas las iteraciones y
entrenamientos necesarios; y el tiempo que requiere cada algoritmo para inferir o llegra a una 
solución para una instancia concreta del problema.\\

En la primera de las comparativas, es evidente que la red neuronal es mucho más lenta y 
computacionalmente intensiva que la optimización clásica. Esto, además de esperado, es lógico, ya 
que la optimizción clásica es un algoritmo iterativo que resuelve el problema de forma directa. Sin
memoria, sin entrenamiento ni aprendizaje, simplemente resuelve el problema de forma
determinista.\\

En cambio, el agente DQN-LSTM requiere un tiempo de entrenamiento inicial mucho mayor, por las 
razones opuestas al algoritmo clásico. La red neuronal sí debe aprender de los datos,
requiere un entrenamiento previo y, por tanto, un tiempo de optimización mucho mayor. Además, con
la dificultad añadida de la incorporación de unas capas de LSTM, cuya función es esencial para el 
más que adecuado funcionamiento de la red, pero incrementa el tiempo de entrenamiento y los recursos
computacionales necesarios.\\

Sin embargo, una vez entrenada, la red neuronal es mucho más capaz que la optimización clásica
para resolver problemas individuales. En este caso, el mejor tiempo de inferencia del agente DQN-LSTM
es incluso menor que el 2,5\% del tiempo de inferencia de la optimización clásica.
Esto se debe a que, una vez entrenada, la red neuronal puede inferir soluciones de forma casi
instantánea, ya que no requiere volver a resolver el problema desde cero, sino que utiliza el
conocimiento adquirido durante el entrenamiento para tomar decisiones rápidas y eficientes.\\

Por tanto, aunque la optimización clásica es más rápida en términos de tiempo total de optimización,
el agente DQN-LSTM es mucho más eficiente en términos de tiempo de inferencia, lo que lo hace
más adecuado para aplicaciones en tiempo real donde se requiere una respuesta rápida y adaptativa.\\